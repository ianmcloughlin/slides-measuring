\documentclass[dvipsnames,hidelinks]{beamer}

  % Enables the use of colour.
  \usepackage{xcolor}
  % Syntax high-lighting for code. Requires Python's pygments.
  \usepackage{minted}
  % Enables the use of umlauts and other accents.
  \usepackage[utf8]{inputenc}
  % Diagrams.
  \usepackage{tikz}
  % Settings for captions, such as sideways captions.
  \usepackage{caption}
  % Symbols for units, like degrees and ohms.
  \usepackage{gensymb}
  % Latin modern fonts - better looking than the defaults.
  \usepackage{lmodern}
  % Allows for columns spanning multiple rows in tables.
  \usepackage{multirow}
  % Better looking tables, including nicer borders.
  \usepackage{booktabs}
  % More math symbols.
  \usepackage{amssymb}
  % More math fonts, like mathbb.
  \usepackage{amsfonts}
  % More math layouts, equation arrays, etc.
  \usepackage{amsmath}
  % More theorem environments.
  \usepackage{amsthm}
  % More column formats for tables.
  \usepackage{array}
  % Adjust the sizes of box environments.
  \usepackage{adjustbox}
  % Better looking single quotes in verbatim and minted environments.
  \usepackage{upquote}
  % Better blank space decisions.
  \usepackage{xspace}
  % Better looking tikz trees.
  \usepackage{forest}
  % URLs.
  \usepackage{hyperref}
  % For plotting.
  \usepackage{pgfplots}
  % For more font sizes.
  \usepackage{anyfontsize}
  
  % Various tikz libraries.
  % For drawing mind maps.
  \usetikzlibrary{mindmap}
  % For adding shadows.
  \usetikzlibrary{shadows}
  % Extra arrows tips.
  \usetikzlibrary{arrows.meta}
  % Old arrows.
  \usetikzlibrary{arrows}
  % Automata.
  \usetikzlibrary{automata}
  % For more positioning options.
  \usetikzlibrary{positioning}
  % Creating chains of nodes on a line.
  \usetikzlibrary{chains}
  % Fitting node to contain set of coordinates.
  \usetikzlibrary{fit}
  % Extra shapes for drawing.
  \usetikzlibrary{shapes}
  % For markings on paths.
  \usetikzlibrary{decorations.markings}
  % For advanced calculations.
  \usetikzlibrary{calc}
  
  % GMIT colours.
  \definecolor{gmitblue}{RGB}{20,134,225}
  \definecolor{gmitred}{RGB}{220,20,60} 
  \definecolor{gmitgrey}{RGB}{67,67,67}
  
  % Change some style options.
  \usetheme{metropolis}
  % Tell minted to use the following colour scheme. 
  \usemintedstyle{manni}

  \metroset{sectionpage=simple, numbering=none, block=fill}

  % Change the default theme colours.
  \setbeamercolor{normal text}{fg=darkgray, bg=white}
  \setbeamercolor{alerted text}{fg=gmitred, bg=white}
  \setbeamercolor{example text}{fg=gmitblue, bg=white}
  \setbeamercolor{frametitle}{fg=white, bg=gmitblue}
  \setbeamercolor*{item}{fg=gmitblue}
  % Use a better math mode font.
  \usefonttheme[onlymath]{serif}

  % \citeurl can be used to a clickable short url to a slide as a reference.
  \renewcommand\footnoterule{}
  \newcommand{\citeurl}[1]{\let\thefootnote\relax\footnotetext{\tiny \textcolor{gmitgrey}{\href{http://#1}{#1}}}}
  \newcommand{\citeeg}[1]{\let\thefootnote\relax\footnotetext{\tiny \textcolor{gmitgrey}{#1}}}
  
  % Prevent minted from showing errors.
  \makeatletter
  \expandafter\def\csname PYGdefault@tok@err\endcsname{\def\PYGdefault@bc##1{{\strut ##1}}}
  \makeatother
  
  \setbeamertemplate{section page}
  {
      \begin{centering}
      \begin{beamercolorbox}[sep=12pt,center]{part title}
      \usebeamerfont{section title}\insertsection\par
      \end{beamercolorbox}
      \end{centering}

      \begin{center}
        \small ian.mcloughlin@gmit.ie
      \end{center}
  } 


\begin{document}
  \section{The Measuring Numbers}

\begin{frame}
  \centering
  \resizebox{0.9\columnwidth}{!}{%
  \begin{tikzpicture}
    \draw[color=gmitred,dotted,-latex] (0,0) arc (180:0:1);
    \draw[latex-latex, color=gmitgrey] (-1.5,0) -- (3.5,0);
    \foreach \x in {-1,0,1,2} \draw[shift={(\x,0)}, color=gmitblue!80] (0pt,0pt) -- (0pt,-5pt) node[below] {\tiny $\x$};
    \draw[shift={(3.14,0)}, color=gmitblue!80] (0pt,0pt) -- (0pt,-5pt) node[below] {\tiny $\pi$};
    \draw[shift={(1.41,0)}, color=gmitblue!80] (0pt,0pt) -- (0pt,-5pt) node[below] {\tiny $\sqrt{2}$};
    \draw[very thick, color=gmitred] (0,0) -- (1,0);
    \end{tikzpicture}
  }
\end{frame}

\begin{frame}[standout]
  \begin{quote}
    What is the distance between 0 and 1? \\[8mm]
    What number comes directly after 0? \\[8mm]
    What number comes directly before 1? \\[8mm]
  \end{quote}
\end{frame}

\begin{frame}
  \begin{columns}
    \begin{column}{0.3\textwidth}
      \color{gmitblue} \fontsize{60}{70}
      \[0.\bar{9}\]
    \end{column}
    {\color{gmitgrey!30}\vrule{}} \hspace{0.01\textwidth}
    \begin{column}{0.5\textwidth}
      \begin{align*}
        &            &  1& &\times && 0.999\ldots & &= & &0&.999\ldots\\[4mm]
        & \Rightarrow& 10& &\times && 0.999\ldots & &= & &9&.999\ldots\\[4mm]
        & \Rightarrow&  9& &\times && 0.999\ldots & &= & &9&.000\ldots\\[4mm]
        & \Rightarrow&  1& &\times && 0.999\ldots & &= & &1&.000\ldots\\[4mm]
        & \Rightarrow&   & &       && 0.999\ldots & &= & &1& \\[4mm]
      \end{align*}
    \end{column}
  \end{columns}
\end{frame}

\begin{frame}
  \begin{columns}
    \begin{column}{0.3\textwidth}
      \color{gmitblue} \fontsize{60}{10}
      \[\mathbb{R}\]
    \end{column}
    {\color{gmitgrey!30}\vrule{}} \hspace{0.1\textwidth}
    \begin{column}{0.5\textwidth}
      \begin{align*}
        &\mathbb{N} &\leftrightarrow& &&\mathbb{R} \\[4mm]
        &1 &\leftrightarrow& &43&.17235759\ldots \\[2mm]
        &2 &\leftrightarrow& &3&.19535351\ldots \\[2mm]
        &3 &\leftrightarrow& &209&.00000000\ldots \\[2mm]
        &4 &\leftrightarrow& &0&.12345678\ldots \\[2mm]
        &\vdots &\leftrightarrow& &&\vdots \\[4mm]
        &{\color{gmitred}?} &{\color{gmitred}\leftrightarrow}& &{\color{gmitred}0}&{\color{gmitred}.2015\ldots}
      \end{align*}
    \end{column}
  \end{columns}
\end{frame}


\begin{frame}[standout]
  \begin{quote}
    Computer programs are countable. \\[8mm]
    Not all numbers are computable.
  \end{quote}
\end{frame} 
\end{document}